\documentclass[11pt, a4paper]{article}

% --- UNIVERSAL PREAMBLE BLOCK ---
\usepackage[a4paper, top=2.5cm, bottom=2.5cm, left=2.5cm, right=2.5cm]{geometry}
\usepackage{fontspec}

\usepackage[english, bidi=basic, provide=*]{babel}

\babelprovide[import, onchar=ids fonts]{english}

% Set default/Latin font to Sans Serif in the main (rm) slot
\babelfont{rm}{Noto Sans}

% --- END UNIVERSAL PREMBLE BLOCK ---

\usepackage{enumitem} % For list formatting
\usepackage{hyperref} % For a modern feel, goes last

% --- DOCUMENT STYLING ---
\title{Detox Jeopardy: How to Play Guide}
\author{}
\date{}

% Remove page numbering
\pagestyle{empty}

% Don't indent paragraphs
\setlength{\parindent}{0pt}
% Add space between paragraphs
\setlength{\parskip}{1em}

% Set up list spacing to be tighter
\setlist{nosep}

% --- DOCUMENT START ---
\begin{document}

\maketitle

Welcome to Detox Jeopardy! This is a fun, two-team game to test your knowledge. Here’s how to play.

\section*{1. Getting Started: Team Setup}

Before you begin, you can customize your teams:

\begin{itemize}
    \item \textbf{Team Names:} Click on "Team 1" or "Team 2" in the white scoreboards to type in your team's name.
    \item \textbf{Team Colors:} Click the small color-picker square next to the team name to choose a color. This color will highlight the scoreboard when it's that team's turn.
\end{itemize}

\section*{2. How to Play: The Game Flow}

Follow these steps to play the game:

\begin{enumerate}
    \item \textbf{Choose a Team:} The game host clicks on a team's scoreboard to make them the "active" team (their scoreboard will light up).
    \item \textbf{Pick a Question:} The active team chooses a category and a dollar amount (e.g., "Coping Skills for \$200"). Click that blue square.
    \item \textbf{Answer the Question:} A pop-up will appear with the question and a 37-second timer.
        \begin{itemize}
            \item The team must answer in the form of a question (e.g., "What is journaling?").
            \item The timer will tick down. You can click the \textbf{Play/Pause button} next to the timer to stop or restart it if needed.
        \end{itemize}
    \item \textbf{Score the Answer:} Once the team answers, the host clicks one of two buttons:
        \begin{itemize}
            \item \textbf{"I Got It Right!":} The team gets the full points. The correct answer will be shown.
            \item \textbf{"I Got It Wrong":} The team \textit{loses half} the points for that question. This also opens up a "steal" opportunity!
        \end{itemize}
    \item \textbf{The Steal! (If a team is wrong):}
        \begin{itemize}
            \item After a wrong answer, the \textit{other} team's scoreboard automatically becomes active.
            \item A new 5-second "steal" timer starts.
            \item The second team can try to answer.
            \item If they are right (click "I Got It Right!"), they get \textit{half} the points.
            \item If they are wrong (click "I Got It Wrong"), no points are lost. The correct answer is then shown.
        \end{itemize}
    \item \textbf{Move On:} After the question is scored, click the \textbf{"Next"} button to close the pop-up.
    \item \textbf{Change Turns:} The game host should now click the \textit{other} team's scoreboard to make it their turn to pick a question.
    \item \textbf{Win the Game:} Repeat until all squares are "ANSWERED". The team with the most points at the end wins!
\end{enumerate}

\section*{3. Special Rule: The Daily Double}

One question on the board is secretly a \textbf{Daily Double}!

\begin{itemize}
    \item When a team picks this question, a special "DAILY DOUBLE" screen will pop up.
    \item Only the team that picked it can answer (no steals!).
    \item The team must \textbf{wager} (or bet) an amount of points \textit{before} seeing the question.
    \item They can wager any amount from \$5 up to their current score (or \$1000, whichever is higher).
    \item If they get the answer right, they \textit{win} their wager. If they get it wrong, they \textit{lose} their wager.
\end{itemize}

\section*{4. How to Play on a Big Screen (TV or Projector)}

To let everyone see the game, you can show it on a big screen from a computer, phone, or tablet.

\subsection*{Method 1: HDMI Cable (Easiest \& Most Reliable)}

\begin{enumerate}
    \item \textbf{You'll Need:} A laptop/computer with an HDMI port and an HDMI cable.
    \item \textbf{Connect:} Plug one end of the HDMI cable into your laptop. Plug the other end into a free HDMI port on the TV or projector.
    \item \textbf{Switch Input:} Use the TV or projector's remote to change the "Input" or "Source" (it might be \texttt{HDMI 1}, \texttt{HDMI 2}, etc.).
    \item \textbf{Done!} Your computer screen is now mirrored on the TV. Play the game on your laptop, and everyone can watch on the big screen.
\end{enumerate}

\subsection*{Method 2: Casting (Wireless)}

This is a good wireless option if you have a Smart TV or a device like a Chromecast, Roku, or Apple TV. Your device and the TV must be on the \textbf{same Wi-Fi network}.

\textbf{From a Computer (Chrome Browser):}
\begin{enumerate}
    \item Open the game in your Chrome browser.
    \item Click the three-dot menu (⋮) in the top-right corner.
    \item Choose \textbf{"Save and share"} and then \textbf{"Cast..."}.
    \item A list of your TVs or casting devices will appear. Click the one you want to use.
\end{enumerate}

\textbf{From a Phone or Tablet:}
\begin{enumerate}
    \item This is usually called \textbf{"Screen Mirroring"}.
    \item \textbf{On an iPhone/iPad:} Swipe down from the top-right corner to open the Control Center. Tap the "Screen Mirroring" icon (two rectangles) and select your Apple TV or compatible Smart TV.
    \item \textbf{On an Android Phone:} Swipe down from the top of the screen. Look for an icon named \textbf{"Cast," "Smart View,"} or \textbf{"Screen Mirroring."} Tap it and select your TV or Chromecast.
    \item Once your screen is mirrored, just open the game, and everyone will be able to see it!
\end{enumerate}

\end{document}